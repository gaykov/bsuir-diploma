\subsection{Разработка даталогической и физической моделей базы данных}
\label{sec:design:db}

Как было упомянуто ранее, в программном средстве, описываемом данным дипломным проектом, будет использоваться специализированная СУБД \nezaboodka. Описание схемы БД для нее производится с помощью специальной нотации. Тем не менее, разрабатывать модель схемы можно с помощью любых средств, однако затем разработанную модель нужно вручную перевести в формат, используемый СУБД. 

На даталогическом уровне модель предметной области представляется в привязке к конкретной СУБД и описывает способ организации данных безотносительно их физического размещения. Описывать модель можно с помощью специальных графических нотаций~\cite{kulikov_db_workbook}. 

Модель даталогического уровня разработаем на основании инфологической модели, описание которой приведено в пункте \ref{sec:domain:model:db}. 

СУБД \nezaboodka разрабатывалась с расчетом на использование программистами в своих приложениях специальной клиентской библиотеки на языке \csharp. В связи с этим большинство стандартных типов данных одноименны представленным в данном ЯП. Кроме того, данная СУБД предлагает уникальный вариант ее использования. Первоначально, по разработанной в специальной нотации схеме БД генерируется исходный код классов модели. Данный исходный код можно включить свой проект и использовать данные классы модели как обычные регулярные классы. Затем, для сохранения экземпляров классов или их последующего поиска и извлечения и используется клиентская библиотека, которая сериализует экземпляры классов при передаче и десериализует их при получении. Таким образом, взаимодействие с данной СУБД несколько отличается от взаимодействия с традиционным SQL решениями. Но это даёт определенные преимущества. Например, появляется возможность использовать сколь угодно сложные структуры данных и легко сохранять и извлекать их. Таким образом, даталогическую модель используемой в разрабатываемом приложении БД можно проектировать с использованием традиционной диаграммы классов \uml.

Описание схемы БД в специальной нотации, используемой в \nezaboodka, приведено в приложении \dbschemeappendix. 

Можно выделить несколько особенностей данной схемы. 

Модель базы данных при разработке подверглась некоторой денормализации. Чаще всего она выражалась в добавлении дополнительных ссылок одних объектов на другие. Причинами такого решения являются следующие:

\begin{itemize}
	\item попытка повысить соответствие модели предметной области;
	\item упрощение манипуляций с данными.
\end{itemize}

Можно заметить, что достаточно часто в качестве типа поля используется тип Byte. Его использование означает, что данные поля смогут принимать только некоторые определенные значения. Вместо использования типов перечисления было принято решение использовать простой целочисленный тип, поскольку СУБД \nezaboodka предоставляет возможность создания специальных справочников. Данная СУБД изначально проектировалась как распределенная, а справочники -- это данные, которые будут храниться на всех узлах.

Необычным для регулярных SQL баз данных является использование списков в качестве типов полей. Как было сказано ранее, возможность из использования появилась вследствие поддержки СУБД сложных агрегирующих типов.

Вследствие этого можно заметить еще одно проявление денормализации: в некоторой дочерней сущности содержится ссылка на родительскую, а родительская содержит список дочерних. Опять же, данное решение вследствие отсутствии необходимости поиска позволит значительно повысить производительность при выборке данных.

Физический уровень моделирования БД описывает конкретные таблицы, связи, индексы, методы хранения, настройки производительности, безопасности. Описывать модель можно с помощью средств, уместных для предметной области~\cite{kulikov_db_workbook}. 

Индексы -- это специальные структуры, применяемые для ускорения операций взаимодействия с данными. В СУБД \nezaboodka они имеют название вторичных индексов. Целесообразно реализовать следующие индексы:

\begin{itemize}
	\item по названиям университетов, административных отделов, факультетов, кафедр, специальностей;
	\item по соответствующим аббревиатурам;
	\item по фамилии и имени пользователей;
	\item по дню недели и времени начала занятий;
	\item по номеру и году поступления групп.
\end{itemize}

Для идентификации используется специальное поле id, которое внедряется во все сущности с помощью наследования от сущности Entity.

Остальные настройки будут применяться непосредственно при развертывании программной системы, поэтому в данном разделе не\linebreakрассматриваются.
